\documentclass[oneside]{book}

\usepackage{amsmath, amsthm, amssymb, amsfonts}
\usepackage{thmtools}
\usepackage{graphicx}
\usepackage{setspace}
\usepackage{geometry}
\usepackage{float}
\usepackage{hyperref}
\usepackage[utf8]{inputenc}
\usepackage[english]{babel}
\usepackage{framed}
\usepackage[dvipsnames]{xcolor}
\usepackage{environ}
\usepackage{tcolorbox}
\newcommand{\bA}{\mathbf{A}}
\newcommand{\bB}{\mathbf{B}}
\newcommand{\bC}{\mathbf{C}}
\newcommand{\bD}{\mathbf{D}}
\newcommand{\bE}{\mathbf{E}}
\newcommand{\bF}{\mathbf{F}}
\newcommand{\bG}{\mathbf{G}}
\newcommand{\bH}{\mathbf{H}}
\newcommand{\bI}{\mathbf{I}}
\newcommand{\bJ}{\mathbf{J}}
\newcommand{\bK}{\mathbf{K}}
\newcommand{\bL}{\mathbf{L}}
\newcommand{\bM}{\mathbf{M}}
\newcommand{\bN}{\mathbf{N}}
\newcommand{\bO}{\mathbf{O}}
\newcommand{\bP}{\mathbf{P}}
\newcommand{\bQ}{\mathbf{Q}}
\newcommand{\bR}{\mathbf{R}}
\newcommand{\bS}{\mathbf{S}}
\newcommand{\bT}{\mathbf{T}}
\newcommand{\bU}{\mathbf{U}}
\newcommand{\bV}{\mathbf{V}}
\newcommand{\bW}{\mathbf{W}}
\newcommand{\bX}{\mathbf{X}}
\newcommand{\bY}{\mathbf{Y}}
\newcommand{\bZ}{\mathbf{Z}}

%% blackboard bold math capitals
\newcommand{\bbA}{\mathbb{A}}
\newcommand{\bbB}{\mathbb{B}}
\newcommand{\bbC}{\mathbb{C}}
\newcommand{\bbD}{\mathbb{D}}
\newcommand{\bbE}{\mathbb{E}}
\newcommand{\bbF}{\mathbb{F}}
\newcommand{\bbG}{\mathbb{G}}
\newcommand{\bbH}{\mathbb{H}}
\newcommand{\bbI}{\mathbb{I}}
\newcommand{\bbJ}{\mathbb{J}}
\newcommand{\bbK}{\mathbb{K}}
\newcommand{\bbL}{\mathbb{L}}
\newcommand{\bbM}{\mathbb{M}}
\newcommand{\bbN}{\mathbb{N}}
\newcommand{\bbO}{\mathbb{O}}
\newcommand{\bbP}{\mathbb{P}}
\newcommand{\bbQ}{\mathbb{Q}}
\newcommand{\bbR}{\mathbb{R}}
\newcommand{\bbS}{\mathbb{S}}
\newcommand{\bbT}{\mathbb{T}}
\newcommand{\bbU}{\mathbb{U}}
\newcommand{\bbV}{\mathbb{V}}
\newcommand{\bbW}{\mathbb{W}}
\newcommand{\bbX}{\mathbb{X}}
\newcommand{\bbY}{\mathbb{Y}}
\newcommand{\bbZ}{\mathbb{Z}}

%% script math capitals
\newcommand{\sA}{\mathscr{A}}
\newcommand{\sB}{\mathscr{B}}
\newcommand{\sC}{\mathscr{C}}
\newcommand{\sD}{\mathscr{D}}
\newcommand{\sE}{\mathscr{E}}
\newcommand{\sF}{\mathscr{F}}
\newcommand{\sG}{\mathscr{G}}
\newcommand{\sH}{\mathscr{H}}
\newcommand{\sI}{\mathscr{I}}
\newcommand{\sJ}{\mathscr{J}}
\newcommand{\sK}{\mathscr{K}}
\newcommand{\sL}{\mathscr{L}}
\newcommand{\sM}{\mathscr{M}}
\newcommand{\sN}{\mathscr{N}}
\newcommand{\sO}{\mathscr{O}}
\newcommand{\sP}{\mathscr{P}}
\newcommand{\sQ}{\mathscr{Q}}
\newcommand{\sR}{\mathscr{R}}
\newcommand{\sS}{\mathscr{S}}
\newcommand{\sT}{\mathscr{T}}
\newcommand{\sU}{\mathscr{U}}
\newcommand{\sV}{\mathscr{V}}
\newcommand{\sW}{\mathscr{W}}
\newcommand{\sX}{\mathscr{X}}
\newcommand{\sY}{\mathscr{Y}}
\newcommand{\sZ}{\mathscr{Z}}


\renewcommand{\emptyset}{\O}

\newcommand{\abs}[1]{\lvert #1 \rvert}
\newcommand{\norm}[1]{\lVert #1 \rVert}
\newcommand{\sm}{\setminus}


\newcommand{\sarr}{\rightarrow}
\newcommand{\arr}{\longrightarrow}

\newcommand{\hide}[1]{{\color{red} #1}} % for instructor version
%\newcommand{\hide}[1]{} % for student version
\newcommand{\com}[1]{{\color{blue} #1}} % for instructor version
%\newcommand{\com}[1]{} % for student version
\newcommand{\meta}[1]{{\color{green} #1}} % for making notes about the script that are not intended to end up in the script
%\newcommand{\meta}[1]{} % for removing meta comments in the script

\DeclareMathOperator{\ext}{ext}
\DeclareMathOperator{\ho}{hole}
\tcbuselibrary{theorems,skins,breakable}

\setstretch{1.2}
\geometry{
    textheight=9in,
    textwidth=5.5in,
    top=1in,
    headheight=12pt,
    headsep=25pt,
    footskip=30pt
}

% Variables
\def\notetitle{Notes on: Introduction to Probability Theory for Economists}
\def\noteauthor{
    \textbf{Enrico Scalas} \\ 
    {\LaTeX} by Agustin Esteva}
\def\notedate{\today}

% The theorem system and user-defined commands
\input{REU-2024/Lectures/theorems}
\input{REU-2024/Lectures/commands}

% ------------------------------------------------------------------------------

\begin{document}
\title{\textbf{
    \LARGE{\notetitle} \vspace*{10\baselineskip}}
    }
\author{\noteauthor}
\date{\notedate}

\maketitle
\newpage

\tableofcontents
\newpage

% ------------------------------------------------------------------------------

\chapter{The Meaning of Probability}
\section{The Classical Definition of Probability}
\defn{Probability}{The probability of an event, $E,$ occurring is defined as \[P(E) = \frac{\text{$\#$ favorable outcomes}}{\text{$\#$ possible outcomes}}\]}
\ex{In coin flip, if $H$ is heads and $T$ is tails, then \[P(H) = \frac{1}{2}\]}

\newpage
\section{Definition, in Practice}
Things get dicey once more events are added into the mix. For example, what is the probability of getting exactly two heads in 3 coin tosses? The event space, with equal probability given to each event, looks like:
\[TTT\quad TTH \quad THH \quad HHH \quad HHT \quad HTT \quad HTH \quad THT\]
so then $P(2H) = \frac{3}{8}$
However, what about with $10$ coin tosses?
\defn{Fundamental Counting Principle}{For a finite sequence of decisions, the number of ways to make these decisions is the product of the number of choices.}
\ex{In the above example, there are $3$ decisions in a sequence (number of coin tosses), and $2$ choices ($T,H$). Thus, there are $2^3 = 8$ total decisions.}

\ex{(Disposition with Repetition) In choosing $N$ objects $n$ times, there are $N^n$ possible choices.}

\ex{(Permutations) By removing $1$ object from $N$ objects until all the objects are selected, to begin with, there are $N$ choices, then $N-1,$ and so on. Thus, there are $N!$ possible decisions.}

\ex{(Disposition without Repetition) In the first, there are $N$ choices, all the way until $N-n+1$ choices, and so the total number of decisions is: $N(N-1) \cdots (N-n+1) = \frac{N!}{n!}$}

\ex{(Combinations) There's a list of $N$ objects and you want to select $n$ object from them, irrespective of the order. \[\binom{N}{n} = \frac{N!}{n!(N-n)!}\]}
\rmkb{Note that $\binom{N}{0} = \binom{N}{N} = 1$}

\ex{(Combination with Repetition) $N$ objects into $n$ boxes, then number of possible choices is $\binom{N+n-1}{N}$}

\ex{(Toin coss, again) What is the probability of getting exactly $n$ heads out of $N$ tosses ($n\leq N.$)? Total number of possible outcomes if $2^N,$ while the number of favorable outcomes is given by choosing $n$ heads out of $N$ possibilities, and so \[P(n-\ext{heads}) = \binom{N}{n}(\frac{1}{2^N})\]}

\newpage
\section{Circularity of Definition}















































































































\end{document}
