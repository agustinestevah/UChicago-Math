\documentclass[oneside]{book}

\usepackage{amsmath, amsthm, amssymb, amsfonts}
\usepackage{thmtools}
\usepackage{graphicx}
\usepackage{setspace}
\usepackage{geometry}
\usepackage{float}
\usepackage{hyperref}
\usepackage[utf8]{inputenc}
\usepackage[english]{babel}
\usepackage{framed}
\usepackage[dvipsnames]{xcolor}
\usepackage{environ}
\usepackage{tcolorbox}
\newcommand{\bA}{\mathbf{A}}
\newcommand{\bB}{\mathbf{B}}
\newcommand{\bC}{\mathbf{C}}
\newcommand{\bD}{\mathbf{D}}
\newcommand{\bE}{\mathbf{E}}
\newcommand{\bF}{\mathbf{F}}
\newcommand{\bG}{\mathbf{G}}
\newcommand{\bH}{\mathbf{H}}
\newcommand{\bI}{\mathbf{I}}
\newcommand{\bJ}{\mathbf{J}}
\newcommand{\bK}{\mathbf{K}}
\newcommand{\bL}{\mathbf{L}}
\newcommand{\bM}{\mathbf{M}}
\newcommand{\bN}{\mathbf{N}}
\newcommand{\bO}{\mathbf{O}}
\newcommand{\bP}{\mathbf{P}}
\newcommand{\bQ}{\mathbf{Q}}
\newcommand{\bR}{\mathbf{R}}
\newcommand{\bS}{\mathbf{S}}
\newcommand{\bT}{\mathbf{T}}
\newcommand{\bU}{\mathbf{U}}
\newcommand{\bV}{\mathbf{V}}
\newcommand{\bW}{\mathbf{W}}
\newcommand{\bX}{\mathbf{X}}
\newcommand{\bY}{\mathbf{Y}}
\newcommand{\bZ}{\mathbf{Z}}

%% blackboard bold math capitals
\newcommand{\bbA}{\mathbb{A}}
\newcommand{\bbB}{\mathbb{B}}
\newcommand{\bbC}{\mathbb{C}}
\newcommand{\bbD}{\mathbb{D}}
\newcommand{\bbE}{\mathbb{E}}
\newcommand{\bbF}{\mathbb{F}}
\newcommand{\bbG}{\mathbb{G}}
\newcommand{\bbH}{\mathbb{H}}
\newcommand{\bbI}{\mathbb{I}}
\newcommand{\bbJ}{\mathbb{J}}
\newcommand{\bbK}{\mathbb{K}}
\newcommand{\bbL}{\mathbb{L}}
\newcommand{\bbM}{\mathbb{M}}
\newcommand{\bbN}{\mathbb{N}}
\newcommand{\bbO}{\mathbb{O}}
\newcommand{\bbP}{\mathbb{P}}
\newcommand{\bbQ}{\mathbb{Q}}
\newcommand{\bbR}{\mathbb{R}}
\newcommand{\bbS}{\mathbb{S}}
\newcommand{\bbT}{\mathbb{T}}
\newcommand{\bbU}{\mathbb{U}}
\newcommand{\bbV}{\mathbb{V}}
\newcommand{\bbW}{\mathbb{W}}
\newcommand{\bbX}{\mathbb{X}}
\newcommand{\bbY}{\mathbb{Y}}
\newcommand{\bbZ}{\mathbb{Z}}

%% script math capitals
\newcommand{\sA}{\mathscr{A}}
\newcommand{\sB}{\mathscr{B}}
\newcommand{\sC}{\mathscr{C}}
\newcommand{\sD}{\mathscr{D}}
\newcommand{\sE}{\mathscr{E}}
\newcommand{\sF}{\mathscr{F}}
\newcommand{\sG}{\mathscr{G}}
\newcommand{\sH}{\mathscr{H}}
\newcommand{\sI}{\mathscr{I}}
\newcommand{\sJ}{\mathscr{J}}
\newcommand{\sK}{\mathscr{K}}
\newcommand{\sL}{\mathscr{L}}
\newcommand{\sM}{\mathscr{M}}
\newcommand{\sN}{\mathscr{N}}
\newcommand{\sO}{\mathscr{O}}
\newcommand{\sP}{\mathscr{P}}
\newcommand{\sQ}{\mathscr{Q}}
\newcommand{\sR}{\mathscr{R}}
\newcommand{\sS}{\mathscr{S}}
\newcommand{\sT}{\mathscr{T}}
\newcommand{\sU}{\mathscr{U}}
\newcommand{\sV}{\mathscr{V}}
\newcommand{\sW}{\mathscr{W}}
\newcommand{\sX}{\mathscr{X}}
\newcommand{\sY}{\mathscr{Y}}
\newcommand{\sZ}{\mathscr{Z}}


\renewcommand{\emptyset}{\O}

\newcommand{\abs}[1]{\lvert #1 \rvert}
\newcommand{\norm}[1]{\lVert #1 \rVert}
\newcommand{\sm}{\setminus}


\newcommand{\sarr}{\rightarrow}
\newcommand{\arr}{\longrightarrow}

\newcommand{\hide}[1]{{\color{red} #1}} % for instructor version
%\newcommand{\hide}[1]{} % for student version
\newcommand{\com}[1]{{\color{blue} #1}} % for instructor version
%\newcommand{\com}[1]{} % for student version
\newcommand{\meta}[1]{{\color{green} #1}} % for making notes about the script that are not intended to end up in the script
%\newcommand{\meta}[1]{} % for removing meta comments in the script

\DeclareMathOperator{\ext}{ext}
\DeclareMathOperator{\ho}{hole}
\tcbuselibrary{theorems,skins,breakable}

\setstretch{1.2}
\geometry{
    textheight=9in,
    textwidth=5.5in,
    top=1in,
    headheight=12pt,
    headsep=25pt,
    footskip=30pt
}

% Variables
\def\notetitle{Notes on: Introduction to Stochastic Processes}
\def\noteauthor{
    \textbf{Gregory Lawler} \\ 
    {\LaTeX} by Agustin Esteva}
\def\notedate{\today}

% The theorem system and user-defined commands
\input{REU-2024/Lectures/theorems}
\input{REU-2024/Lectures/commands}

% ------------------------------------------------------------------------------

\begin{document}
\title{\textbf{
    \LARGE{\notetitle} \vspace*{10\baselineskip}}
    }
\author{\noteauthor}
\date{\notedate}

\maketitle
\newpage

\tableofcontents
\newpage

% ------------------------------------------------------------------------------

\chapter{Preliminaries}
\section{Introduction}
\defn{Stochastic Process}{A \textit{stochastic process} is a collectino of random variables $X_t$ indexed by time. In other words, it is a random process evolving with time}
\rmkb{The random variables $X_t$ take values in a set called the \textit{state space}.}

\defn{Markov Processes}{A \textit{Markov process} is a stochastic process in which the change at time $t$ is determined by the value of the process at time $t$ and not by the values before $t.$}

\section{Linear Differential Equations}
Consider the homogeneous differential equation,
\[y^{(n)}(t) + a_{n-1}y^{(n-1)}(t) + \dots + a_1y'(t) + a_0y(t) = 0\] where $a_0, a\dots, a_{n-1}$ are constant. For any initial conditions of \[y(0) = b_0, \; y'(0) = b_1, \dots, y^{(n-1)}(0) = b_{n-1},\] there exists a unique solution satisfying these conditions. Given that $y_1(t), \dots, y_n(t)$ are linearly independent solutions to the above, then every solution can be written in the form \[y(t) = c_1y_1(t) + \dots + c_ny_n(t)\] for constants $c_1, \dots, c_n.$
The solutions $y_1, \dots, y_n$ are found by looking for solutions of the form $y(t) = e^{\lambda t}.$ Thus,
\[\lambda^n + a_{n-1}\lambda^{n-1} + \dots + a_1\lambda + a_0 = 0\]
If the polynomial has $n$ distinct roots, $\lambda_1, \dots, \lambda_n,$ then there are $n$ linearly independent solutions $e^{\lambda_1t}, \dots, e^{\lambda_nt}.$ If there are repeated roots, i.e, some $\lambda$ has multiplicity $j,$ then $e^{\lambda t}, te^{\lambda t}, \dots, t^{j-1}e^{\lambda t}.$ Therefore, each root with multiplicity $j$ has $j$ linearly independent solutions. Now consider
\begin{align*}
y'_1(t) &= a_{11}y_1(t) + a_{12}y_2(t) + \dots + a_{1n}y_n(t)    \\
y'_2(t) &= a_{21}y_1(t) + a_{22}y_2(t) + \dots + a_{2n}y_n(t)    \\
\vdots &\qquad \qquad \qquad \vdots\\
y'_n(t) &= a_{n1}y_1(t) + a_{n2}y_2(t) + \dots + a_{nn}y_n(t)
\end{align*}
or, equivalently, \[\overline{y}'(t) = \textbf{A}\overline{y}(t).\]
There is a unique solution satisfying $\overline{y}(0) = \textbf{v}.$ This solution can be written in the form
\[\overline{y}(t) = e^{t\textbf{A}}\overline{v}\] which can be defined in terms of a power series, \[e^{t\textbf{A}} = \sum_{j = 0}^\infty \frac{(t\textbf{A})^j}{j!}\] and thus if $D = \begin{bmatrix}
    d_1 & 0 & \dots & 0\\
    0 & d_2 & \dots & 0\\
    \vdots & \vdots & \ddots & \vdots \\
    0 & 0 & \dots & d_n
\end{bmatrix}$ is the diagonal matrix of $\textbf{A} = \textbf{Q}^{-1}\textbf{D}\textbf{Q},$ then \[e^{t\textbf{A}} = \textbf{Q}^{-1}e^{t\textbf{D}}\textbf{Q} = \textbf{Q}^{-1}\begin{bmatrix}
    e^{td_1} & 0 & \dots & 0\\
    0 & e^{td_2} & \dots & 0\\
    \vdots & \vdots & \ddots & \vdots \\
    0 & 0 & \dots & e^{td_n}
\end{bmatrix}\textbf{Q}\]
\section{Linear Difference Equations}
Consider the equation
\[f(n) = af(n-1) + bf(n+1), \qquad K< n< N\]
If $f$ satisfies the above and $f(K), f(K+1)$ are known, then $f(n)$ can be determined for all $K \leq n \leq N$ recursively by the formula:
\[f(n+1) = \frac{1}{b}[f(n) - af(n-1)].\]


Conversely, if $u_0,U_1$ are any real numbers, we can find a solution satisfying $f(K) = u_0, f(K+1) = u_1$ by defining $f(n)$ recursively as above. Note that the set of solutions satisfying the above is a vector space with dimension $2,$ where the basis is given by $\beta = \{f_1, f_2\},$ where $f_1(K) = 1\qquad f_1(K+1) = 0$ and $f_2(K) = 0 \qquad f_2(K+1) = 1.$ We will try to find a pair of such linearly independent solutions, first of the form $f(n) = \alpha^n$ for some $\alpha \neq 0.$ $f$ is a solution for $\alpha$ if an only if \[\alpha^n = a\alpha^{n-1} + b\alpha^{n+1}\qquad K < n< N\] which implies that \[\alpha = a + b\alpha^2\] and thus we can solve the quadratic equation \begin{align}
    \alpha = \frac{1 \pm \sqrt{1 - 4ab}}{2b}
\end{align}
\begin{enumerate}
    \item If $1-4ab\neq 0,$ then $\alpha_1,\alpha_2$ are two distinct roots to the equation, and so the general solution is \[f(n) = c_1 \alpha_1^n + c_2 \alpha_2^n.\]
    \item If $1-4ab = 0,$ then there exists only one solution $g_1(n) = \alpha^n = (\frac{1}{2b})^n.$ If $g_n(n) = n(\frac{1}{2b})^n,$ we can see that \[ag_2(n-1) + bg_2(n+1) = g_2(n)\] And thus $g_2$ is a solution.
\end{enumerate}

\ex{Suppose we want to find a function $f$ satisfying $f(n) = \frac{1}{6}f(n-1) + \frac{2}{3}f(n+1), \qquad 0 < n < \infty$} with \[f(0) = 4\qquad f(1) = 3.\] Plugging into (1.1), we find that $\alpha = \frac{3 \pm 5}{4},$ and thus the general solution is:
\[f(n) = c_1 \left(\frac{3 + 5}{4}\right)^n + c_2\left(\frac{3 - 5}{4}\right)^n\]












































































































\end{document}
